
\documentclass[12pt,a4paper]{article}

\usepackage{amssymb}
% Cases à cocher personnalisées
\newcommand{\CheckedBox}{\ensuremath{\boxtimes}} % case "cochée"
\newcommand{\EmptyBox}{\ensuremath{\square}} % case "non cochée"
\usepackage{graphicx}
\usepackage[a4paper, top=3cm, bottom=1.8cm, left=1.2cm, right=1.2cm]{geometry}
\usepackage{titlesec}
\usepackage{titleps}

\newpagestyle{pontsstyle}{
    \sethead{}{\textbf{École nationale des Ponts et Chaussées}}{}
    \setfoot{}{}{}
}

\pagestyle{pontsstyle}
\setlength{\headsep}{1.5cm}
\usepackage[utf8]{inputenc}
\usepackage[T1]{fontenc}
\usepackage[french]{babel}
\usepackage{longtable}

\setlength{\parindent}{0pt}
\renewcommand{\arraystretch}{1.2}

\begin{document}

\begin{center}
    {\Large \textbf{Annuel des PFE - TEST 4}}\\[0.2cm]
    Merci de respecter la casse (minuscules/majuscules).
\end{center}

\vspace{0.7cm}

\begin{longtable}{|p{7cm}|p{9cm}|}
\hline
    \textbf{Département d'enseignement} & Génie Civil \\ \hline
    \textbf{Prénom NOM} & JAKOB DUPONT \\ \hline
    \textbf{Adresse mail permanente} & Vincent.Cannizzaro@example.com \\ \hline
    \textbf{Statut étudiant entrepreneur} & $\Box$~ \\ \hline
    \textbf{Profil (voie d'entrée)} & Voie classique \\ \hline
    \textbf{Si Double diplôme, préciser Établissement, Ville, Pays} & Université X, Ville Y, Pays Z \\ \hline
    \textbf{Titre du parcours ou de toute formation diplômante en 3A (le cas échéant), qu'elle soit effectuée dans l'École ou à l'extérieur} & Master en Transport \\ \hline
    \textbf{Si vous avez indiqué une formation diplômante à la case précédente, précisez le nom de l'établissement, la ville et le pays.} & Université X, Ville Y, Pays Z \\ \hline
    \textbf{Promotion (indiquez l'année)} & 2023 \\ \hline
    \textbf{Votre photo portrait (minimum 295 pixels de large, sous format jpeg ou png)} & \center\includegraphics[width=4cm]{photo.jpg} \\ \hline
    \textbf{Type de PFE} & Recherche \\ \hline
    \textbf{Organisme du PFE} & Entreprise ABC \\ \hline
    \textbf{Type d'organisme d'accueil} & Industrie \\ \hline
    \textbf{Tuteur professionnel} sous la forme Prénom NOM& Marie DURAND \\ \hline
    \textbf{Fonction du tuteur professionnel} & Ingénieure R\&D \\ \hline
    \textbf{Tuteur académique} sous la forme Prénom NOM & Luc MARTIN \\ \hline
    \textbf{Fonction du tuteur académique} & Maître de conférences \\ \hline
    \textbf{Organisme de rattachement du tuteur académique} & ENPC \\ \hline
    \textbf{Langue de rédaction du PFE} & Français \\ \hline
    \textbf{Si PFE Non confidentiel} & ~Je suis en accord avec la phrase suivante :~$\CheckedBox$~ \newline\newline« Pour l'Annuel des PFE, je fournis dans ce questionnaire des éléments non confidentiels de mon travail. J'en autorise la diffusion au format numérique et papier. »\\ \hline
    \textbf{Si PFE Confidentiel} &  Je suis d'accord avec la phrase suivante : ~$\CheckedBox$~ \newline\newline « Je suis informé(e) que les éléments textuels et iconographiques demandés dans ce questionnaire sont destinés à une publication numérique et papier (Annuel des PFE). Je m'engage à ne fournir que des éléments diffusables et en autorise la diffusion. » \\ \hline
    \textbf{Durée de confidentialité} & 0 mois \\ \hline
    \textbf{Titre du PFE en français} (Entre 10 et 75 signes espaces compris) & Mon PFE génial \\ \hline
    \textbf{Thématique principale \newline\newline 
(Se référer à la liste fournie par le Département)} & Transport \\ \hline
    \textbf{Mots-clés en français (3 à 5) (séparés par des ; sans majuscule)} & super; pfe; intéressant \\ \hline
    \textbf{Présentation du contexte du PFE (version non confidentielle)} Longueur conseillée entre 200 et 450 signes, espaces compris
 & 13 caractères13 caractères13 caractères13 caractères13 caractères13 caractères13 caractères13 caractères13 caractères13 caractères13 caractères13 caractères13 caractères13 caractères13 caractères13 caractères \\ \hline
    \textbf{Présentation des missions et objectifs du PFE (version non confidentielle)} Longueur conseillée entre 200 et 450 signes, espaces compris & 13 caractères13 caractères13 caractères13 caractères13 caractères13 caractères13 caractères13 caractères13 caractères13 caractères13 caractères13 caractères13 caractères13 caractères13 caractères13 caractères13 caractères13 caractères13 caractères13 caractères \\ \hline
    \textbf{Résumé en français du PFE (version non confidentielle)} Longueur conseillée entre 1150 et 1700 signes, espaces compris & 13 caractères13 caractères13 caractères13 caractères13 caractères13 caractères13 caractères13 caractères13 caractères13 caractères13 caractères13 caractères13 caractères13 caractères13 caractères13 caractères13 caractères13 caractères13 caractères13 caractères13 caractères13 caractères13 caractères13 caractères13 caractères13 caractères13 caractères13 caractères13 caractères13 caractères13 caractères13 caractères13 caractères13 caractères13 caractères13 caractères13 caractères13 caractères13 caractères13 caractères13 caractères13 caractères13 caractères13 caractères13 caractères13 caractères13 caractères13 caractères13 caractères13 caractères13 caractères13 caractères13 caractères13 caractères13 caractères13 caractères13 caractères13 caractères13 caractères13 caractères13 caractères13 caractères13 caractères13 caractères13 caractères13 caractères13 caractères13 caractères13 caractères13 caractères13 caractères13 caractères13 caractères13 caractères13 caractères13 caractères13 caractères13 caractères13 caractères13 caractères13 caractères13 caractères13 caractères13 caractères13 caractères13 caractères13 caractères13 caractères13 caractères13 caractères13 caractères13 caractères13 caractères13 caractères13 caractères13 caractères13 caractères13 caractères13 caractères13 caractères \\ \hline
    \textbf{Titre du PFE en anglais} (Entre 10 et 75 signes espaces compris) & Mon PFE cool \\ \hline
    \textbf{Mots-clés en anglais (3 à 5)} & super; pfe; intéressant;génial \\ \hline
    \textbf{Présentation du contexte du PFE en anglais (version non confidentielle)} Longueur conseillée : entre 200 et 450 signes, espaces compris & 13 caractères13 caractères13 caractères13 caractères13 caractères13 caractères13 caractères13 caractères13 caractères13 caractères13 caractères13 caractères13 caractères13 caractères13 caractères13 caractères \\ \hline
    \textbf{Présentation des missions et objectifs du PFE en anglais (version non confidentielle)} Longueur conseillée : entre 200 et 450 signes, espaces compris & 13 caractères13 caractères13 caractères13 caractères13 caractères13 caractères13 caractères13 caractères13 caractères13 caractères13 caractères13 caractères13 caractères13 caractères13 caractères13 caractères13 caractères13 caractères13 caractères13 caractères \\ \hline
    \textbf{Résumé du PFE en anglais (version non confidentielle)}\newline\newline Longueur conseillée entre 1150 et 1700 signes, espaces compris & 13 caractères13 caractères13 caractères13 caractères13 caractères13 caractères13 caractères13 caractères13 caractères13 caractères13 caractères13 caractères13 caractères13 caractères13 caractères13 caractères13 caractères13 caractères13 caractères13 caractères13 caractères13 caractères13 caractères13 caractères13 caractères13 caractères13 caractères13 caractères13 caractères13 caractères13 caractères13 caractères13 caractères13 caractères13 caractères13 caractères13 caractères13 caractères13 caractères13 caractères13 caractères13 caractères13 caractères13 caractères13 caractères13 caractères13 caractères13 caractères13 caractères13 caractères13 caractères13 caractères13 caractères13 caractères13 caractères13 caractères13 caractères13 caractères13 caractères13 caractères13 caractères13 caractères13 caractères13 caractères13 caractères13 caractères13 caractères13 caractères13 caractères13 caractères13 caractères13 caractères13 caractères13 caractères13 caractères13 caractères13 caractères13 caractères13 caractères13 caractères13 caractères13 caractères13 caractères13 caractères13 caractères13 caractères13 caractères13 caractères13 caractères13 caractères13 caractères13 caractères13 caractères13 caractères13 caractères13 caractères13 caractères13 caractères13 caractères13 caractères \\ \hline
    \textbf{Bibliographie}\newline\begin{itemize} \item Référence 1 (optionnel) : \item Référence 2 (optionnel) : \item Référence 3 (optionnel) : \end{itemize} &\begin{itemize} \item Réf.1  \item Réf.2  \item Réf.3 \end{itemize}  \\ \hline
    \textbf{Image associée} \newline\newline
    (Graphique, photographie…)\newline\newline
    \textbf{À noter :} Sans références précises l’image ne sera pas publiée.\newline\newline
    Si vous n’êtes pas l’auteur principal de l’illustration utilisée, vous pouvez demander au 
    propriétaire une autorisation d’utilisation et de reproduction. Si vous l’obtenez, merci de
    la joindre à votre dossier.& Nom du fichier transmis en annexe de ce fichier 
    Word, au format jpeg ou png : \newline\newline  image.jpg \newline\newline \textbf{Légende :}\newline\newline Legende de l'image
    \newline\newline Droits : Nom du photographe ou de l’auteur, références bibliographiques d’où est issue l’image…\newline\newline
    Meow\\ \hline
\end{longtable}
\vspace{0.8cm}
\newpage
% Cases à cocher contrôlées par Python
\noindent $\CheckedBox$~Je confirme que les éléments textuels et iconographiques transmis dans ce questionnaire
sont non confidentiels et diffusables au format numérique et papier (Annuel des PFE).\\[0.25cm]

\noindent $\CheckedBox$~Je confirme et autorise la diffusion.\\[0.25cm]

\noindent $\CheckedBox$~Je certifie sur l'honneur l'exactitude des informations.

\vspace{1cm}

\begin{center}
\rule{4cm}{0.4pt}\\

\end{center}

\vspace{0.8cm}

\textbf{Validation du département}

\vspace{0.2cm}

\begin{tabular}{|p{7cm}|p{9cm}|}
\hline
\textbf{Date de validation} & \\[0.8cm] \hline
\textbf{Nom du valideur} & \\[0.8cm] \hline
\textbf{Mail du valideur} & \\[0.8cm] \hline
\end{tabular}

\end{document}
